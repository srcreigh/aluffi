
% Solutions to exercises in Paolo Aluffi's "Algebra: Chapter 0"
% All solutions copyright 2015 Shane Creighton-Young

% The scrartcl document class lets us have a wider stroke width for the fonts. However, it
% changes the heading font by default so we've used the \setkomafont command to
% restore the headings to the vanilla LaTeX style.
\documentclass[fontsize=14pt]{scrartcl}
\setkomafont{disposition}{\normalfont\bfseries}

% Standard American Math Society packages provide theorem environments, symbols,
% etc.
\usepackage{amsthm}
\usepackage{amsmath}
\usepackage{amssymb}

% Add lcm operator.
\DeclareMathOperator{\lcm}{lcm}

% The tiks-cd package provides macros for easily writing commutative diagrams.
% For more info see [1].
% [1]: http://ctan.math.ca/tex-archive/graphics/pgf/contrib/tikz-cd/tikz-cd-doc.pdf
\usepackage{tikz-cd}

% The pdfrender package is used to output fonts with a larger stroke width.
\usepackage{pdfrender}

% The mathrsfs package provides the script font.
\usepackage{mathrsfs}

\usepackage[margin=0.6in]{geometry}

% The chngcntr ("change counter") package is used here so that subsection
% numbers are written without the leading section number. This takes place in
% the subsection headings as well as the theorem environment numbering.
%
% Before:
% 1. Section
% 1.1. Subsection
% Problem 1.1.1. What is 1 + 1?
% Problem 1.1.2. What is 1 + 2?
%
% After:
% 1. Section
% 1. Subsection
% Problem 1.1. What is 1 + 1?
% Problem 1.2. What is 1 + 2?
\usepackage{chngcntr}
\counterwithout{subsection}{section}

% The problem environment is a regular ams theorem environment with "Problem"
% text and some leading space to give some separation between the problems.
\theoremstyle{definition}
\newtheorem{problem-internal}{Problem}[subsection]
\newenvironment{problem}{
  \medskip
  \begin{problem-internal}
}{
  \end{problem-internal}
}

% The solution environment is a proof environment with the "solution" text as
% well as the following adjustments:
% - No indent on paragraphs;
% - A small amount of space between paragraphs.
%
% Note: The negative space at the beginning is to remove the space before the
% first paragraph in the solution.
\newenvironment{solution}{
  \begin{proof}[Solution]
  \vspace{-8px}
  \setlength{\parskip}{4px}
  \setlength{\parindent}{0px}
}{
  \end{proof}
}

% Renewing the \thesection command changes the section numbers to roman
% numerals. This matches the style of the Aluffi textbook.
%
% Before:
% 1. Section
% 1.1. Subsection
%
% After:
% I. Section
% I.1. Subsection
\renewcommand{\thesection}{\Roman{section}} 

% Renewing the \qedsymbol command changes the QED symbol. This QED symbol is
% automatically used by the proof environment.

% Set with spacing padding for the curly braces.
\newcommand{\set}[1]{\left\{\,#1\,\right\}}
\newcommand{\id}{\mathrm{id}}
\newcommand{\im}{\mathrm{im}}
\newcommand{\Obj}{\mathrm{Obj}}
\newcommand{\Hom}{\mathrm{Hom}}
\newcommand{\abs}[1]{\left|#1\right|}
% Inuitive "from" command draws an arrow pointing left, A <- B reads "A from p"
\newcommand{\from}{\leftarrow}
\newcommand{\quotuniv}[1]{\overline{#1}}
% inverses should have a -1
\newcommand{\inv}[1]{#1^{-1}}

\begin{document}



\section*{Chapter 2}
\subsection*{Groups: First encounter}
\setcounter{subsection}{1}


% Problem 1.1
\begin{problem}
Write a careful proof that every group is the group of isomorphisms of a
groupoid. In particular, every group is the group of automorphisms of some
object in some category. (\S2.1]
\end{problem}
\begin{solution}
\def \C {\mathsf{C}}
Let $G$ be a group with binary operation $\circ$ and identity $e$. Consider a
category $\C$ with a single object $\emptyset$. We will show that the elements
of $G$ make suitable morphisms $\Hom_\C(\emptyset,\emptyset)$ with $\circ$ as a
composition operation.

Since $G$ is a group, $G$ (i.e. $\Hom_\C(\emptyset,
\emptyset)$) is closed $\circ$ and $\circ$ is transitive for morphisms. The
identity $e$ makes an appropriate identity morphism for $\emptyset$. Hence $\C$
is a category.

Lastly, we will show that $\C$ is a groupoid. Any morphism
$f\in\Hom_\C(\emptyset, \emptyset)$ has a (two-sided) inverse $f^{-1}$ since $G$
is a group. This means that $f$ is an isomorphism.
\end{solution}


% Problem 1.2
\begin{problem}
Consider the `sets of numbers' listed in \S1.1, and decide which are made into
groups by conventional operations such as $+$ and $\cdot$. Even if the answer is
negative (for example, $(\mathbb{R},\cdot)$ is not a group), see if variations
on the definition of these sets lead to groups (for example,
$(\mathbb{R}^*,\cdot)$ is a group; cf. \S1.4). [\S1.2]
\end{problem}
\begin{solution}
This is open-ended, so I don't want to do it.
\end{solution}

\begin{problem}
Prove that $\inv{(gh)} = \inv{h}\inv{g}$ for all elements $g,h$ of a group $G$.
\end{problem}
\begin{solution}
Let $B$ be a group and suppose $f,g\in G$. Then we have:
%
\[ (\inv{g}\inv{f})(fg) = (\inv{g}(\inv{f}f))g = (\inv{g}e)g = \inv{g}g = e \]
\[ (fg)(\inv{g}\inv{f}) = (f(g\inv{g}))\inv{f} = (fe)\inv{f} = f\inv{f} = e \]
%
Hence $\inv{g}\inv{f}$ is a two-sided inverse of $gf$.
\end{solution}


% Problem 1.3
\begin{problem}
Suppose that $g^2 = e$ for all elements $g$ of a group $G$; prove that $G$ is
commutative.
\end{problem}
\begin{solution}
Let $G$ be a group with $g,h\in G$ such that $g^2=h^2=e$. Then,
%
\[ gh = eghe = hhghgg = h(hg)^2g = hg \]
%
as required.
\end{solution}


% Problem 1.4
\begin{problem}
The `multiplication table' of a group is an array compiling the results of all
multiplications $g\bullet h$:

[Redacted.]

(Here $e$ is the identity element. Of course the table depends on the order in
which the elements are listed in the top row and leftmost column.) Prove that
every row and every column of the multiplication table of a group contains all
elements of the group exactly once (like Sudoku diagrams!).
\end{problem}
\begin{solution}
Without loss of generality suppose that two elements in a column are different,
i.e. for some fixed element $f$ we have $f\bullet g = f\bullet h$. Then by
(left-)cancellation we get that $g=h$. Hence the columns must be the same.
\end{solution}


% Problem 1.5
\begin{problem}
$\neg$ Prove that there is only one possible multiplication table for $G$ if $G$
has exactly 1, 2, or 3 elements. Analyze the possible multiplication tables for
groups with exactly 4 elements, and show that there are two distinct tables, up
to reordering the elements of $G$. Use these tables to prove that all groups
with $< 4$ elements are commutative.

(You are welcome to analyze groups with 5 elements using the same technique, but
you will soon know enough about groups to be able to avoid such brute-force
approaches.) [2.19]
\end{problem}
\begin{solution}
(Note: I spent a lot of time trying to figure out why there seemed to be many
more than 2 tables for groups with exactly 4 elements until going back to the
question where it there are only 2 ``up to reordering''!)

If a group only has one element, say $G=\set{e}$, there is only one spot to fill.
Since multiplication must be closed it must be $e$. Here $e$ is the identity.

If a group has two elements, say $G=\set{e,g}$, then there are four spots to
fill. The $e$-row and $e$-column are easy, so there is only one non-trivial
column to fill. If $gg=g$ then by cancellation we get that $g=e$, which is a
contradiction since $g$ is distinct from $e$. So $gg=e$.

Suppose that $G=\set{e,g,h}$. Here there are four non-trivial spots to fill.  We
can't have that $gh=h$ (by cancellation as above), so we must have $gh=e$. For
the same reason, we must have $hg=e$. There is no other choice but to set
$g^2=h$ and $h^2=g$.

Suppose that $G=\set{e,f,g,h}$. This is the tricky one. There are nine spots to
fill. First of all, we note that due to the double-sidedness of inverses, we
have that for $f,g,h\in G$ (all distinct) we can't have that $fg=h$ and $gf=e$
(the second implies that $f=\inv{g}$, but then the first says $\inv{g}{g}\neq
e$, a contradiction.) This in fact implies commutativity, since for any distinct
$f,g\in G$ it must either be the case that $fg=e=gf$ or $fg=h=gf$.

The question states that there are only two ``up to the order of the elements''.
So, rather than answer ``how many possible tables are there'', we will answer:
how many $n$ are such that $n$ elements of $G$ go to the identity when squared?
We'll see that only $n=1$ and $n=3$ work; and that the only differences in the
tables for $n=1$ are the choosing of which element has order 2 (i.e. which
element goes to the identity when it is squared;) and that there is only one
table where $n=3$.

Clearly it can't work for $n<0$ or $n>3$ (we can't choose less than 0 or more
than 3 elements from $G$.) We claim that it can't be that no element in $G$ has
order 2. To see this, without loss of generality consider $f$. We know that $f$
must have an inverse that is distinct from itself (or else we get an immediate
contradiction) and from $e$---without loss of generality, suppose $\inv{f}=h$.
This also means that $\inv{h}=f$.  Now we ask: which element can be the inverse
of $g$? It can't be $f$ or $h$ because the inverse is unique and $g$ is distinct
from $f$ and $h$. It also can't be $e$ since $ge=g$ by definition.  So it must
be $g$. But this means that there are 1 elements that have order 2.  We have
shown that, if any element in $G$ doesn't have order 2, another must have order
2. So zero order two elements is a contradiction.

Now we claim that having only two elements of order not equal to 2 is also
infeasible. Assume without loss of generality that $f^2=e$ and $g^2=e$. What is
$h^2$? We note that the inverse of $h$ cannot be $e$ by cancellation, and cannot
be $f$ or $g$ since inverses are unique and $h$ is distinct from $f$ and $g$.
This means that $h$ must be its own inverse, and thus has order 2. Therefore,
two elements of order 2 implies three elements of order 2.

Now we'll show that one order 2 element works okay. Without loss of generality
assume $f^2=e$. Here we must take $gh=hg=e$. These are three cells in our table.
Furthermore, we must have $fg=h$ (since $fg$ can't be $f$ or $g$ or $e$), and
similarly for $fh=g$. By commutativity we also have that $gf=h$ and $hf=g$. Now
we've filled seven cells in our table. By our above ``theorem'' about column and
row uniqueness we know we must take $gh=hg=f$. Hence there is only one group of
size 4 with only one order 2 element (up to the selection of which element has
order 2.)

Last, consider the group of size 4 where all elements are order 2. Immediately
we deducate that $fg=h$ (since it can't be $e$, $f$, or $g$ by cancellation,)
and then that $fh=g$ (since it is the last option!) Similarly that $gh=f$, and
so on---there is only one group where all elements are order 2.
\end{solution}


\end{document}
